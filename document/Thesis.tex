\documentclass[12pt,oneside]{uhthesis}
\usepackage{subfigure}
%\usepackage[ruled,lined,linesnumbered,titlenumbered,algochapter,spanish,onelanguage]{algorithm2e}
\usepackage{amsmath}
\usepackage{amssymb}
\usepackage{amsbsy}
\usepackage{caption,booktabs}
\captionsetup{ justification = centering }
%\usepackage{mathpazo}
\usepackage{float}
\setlength{\marginparwidth}{2cm}
\usepackage{todonotes}
\usepackage{listings}
\usepackage{xcolor}
\usepackage{multicol}
\usepackage{graphicx}
\floatstyle{plaintop}
\restylefloat{table}
\addbibresource{Bibliography.bib}
% \setlength{\parskip}{\baselineskip}%
\renewcommand{\tablename}{Tabla}
%\renewcommand{\listalgorithmcfname}{Índice de Algoritmos}
%\dontprintsemicolon
%\SetAlgoNoEnd

\definecolor{codegreen}{rgb}{0,0.6,0}
\definecolor{codegray}{rgb}{0.5,0.5,0.5}
\definecolor{codepurple}{rgb}{0.58,0,0.82}
\definecolor{backcolour}{rgb}{0.95,0.95,0.92}

\lstdefinestyle{mystyle}{
    backgroundcolor=\color{backcolour},   
    commentstyle=\color{codegreen},
    keywordstyle=\color{purple},
    numberstyle=\tiny\color{codegray},
    stringstyle=\color{codepurple},
    basicstyle=\ttfamily\footnotesize,
    breakatwhitespace=false,         
    breaklines=true,                 
    captionpos=b,                    
    keepspaces=true,                 
    numbers=left,                    
    numbersep=5pt,                  
    showspaces=false,                
    showstringspaces=false,
    showtabs=false,                  
    tabsize=4
}
 
\lstset{style=mystyle}

\title{Plataforma de Inteligencia de Negocios para el análisis comercial de la Red Minorista en el Grupo Empresarial CIMEX}
\author{\\\vspace{0.25cm}Nombre de los autores\\\vspace{0.2cm}Nombre del segundo tutor}
\advisor{\\\vspace{0.25cm}Nombre del primer tutor\\\vspace{0.2cm}Nombre del segundo tutor}
\degree{Licenciado en Ciencia de la Computación}
\faculty{Facultad de Matemática y Computación}
\date{La Habana, noviembre 2022\\\vspace{0.25cm}\href{https://github.com/Thesis-SAN}{https://github.com/Thesis-SAN}}
\logo{Graphics/uhlogo}
\makenomenclature

\renewcommand{\vec}[1]{\boldsymbol{#1}}
\newcommand{\diff}[1]{\ensuremath{\mathrm{d}#1}}
\newcommand{\me}[1]{\mathrm{e}^{#1}}
\newcommand{\pf}{\mathfrak{p}}
\newcommand{\qf}{\mathfrak{q}}
%\newcommand{\kf}{\mathfrak{k}}
\newcommand{\kt}{\mathtt{k}}
\newcommand{\mf}{\mathfrak{m}}
\newcommand{\hf}{\mathfrak{h}}
\newcommand{\fac}{\mathrm{fac}}
\newcommand{\maxx}[1]{\max\left\{ #1 \right\} }
\newcommand{\minn}[1]{\min\left\{ #1 \right\} }
\newcommand{\lldpcf}{1.25}
\newcommand{\nnorm}[1]{\left\lvert #1 \right\rvert }
\renewcommand{\lstlistingname}{Ejemplo de código}
\renewcommand{\lstlistlistingname}{Ejemplos de código}

\begin{document}

\frontmatter
\maketitle

\include{FrontMatter/Dedication}
\include{FrontMatter/Thanks}
\include{FrontMatter/SupervisorOpinion}
\include{FrontMatter/Abstract}
\include{FrontMatter/Contents}

\mainmatter

\chapter*{Introducción}\label{chapter:introduction}
\addcontentsline{toc}{chapter}{Introducción}


Los múltiples avances tecnológicos y técnicos, como herramientas para el desarrollo del conocimiento, han permitido a las organizaciones el manejo inteligente y óptimo de sus recursos, les ha dado acceso a elementos y prácticas innovadoras que permiten poner en marcha estrategias y acciones efectivas para manejar sus negocios de manera organizada y mantener un riguroso control operacional. Las empresas cubanas promueven la explotación de estas ventajas brindadas por el ascenso tecnológico.\\

La Corporación CIMEX, es un grupo empresarial, representado  a lo largo de todo el territorio nacional, con actividades en importantes sectores de la economía cubana, estructurado en 19 sucursales territoriales, más de 80 empresas, 10 subsidiarias y 13 divisiones especializadas comprenden amplios e importantes sectores de la economía cubana. Una de sus estructuras más significativa es la Red de Tiendas Minoristas, integrada por las Tiendas Panamericanas, los Servicentros “Servicupet”, las Cafeterías “El Rápido”, los Videocentros, las Tiendas Fotográficas “Photoservice”, los Centros Comerciales y pequeñas tiendas típicas en barrio (Corporación CIMEX - Ecured, 2012).\\

La Corporación CIMEX surge en el año 1979, con la creación de su primera división, Havanatur, encargada del movimiento de personas entre Cuba y Estados Unidos. En una segunda fase de desarrollo, comprendida  hasta 1992, crea otras divisiones con actividades de perfil comercial, fundamentalmente, de carácter mayorista. En 1993, a partir de la despenalización de la tenencia de divisas, el auge del turismo internacional y el inicio de la recuperación económica del país, CIMEX inicia una etapa de expansión al sector de los servicios minoristas, ampliándose su objeto social, que comprende hoy,  la realización de actividades productivas, de carácter comercial mayorista, minorista y de servicios, la exportación y la importación; constituyendo su rasgo más significativo el amplio espectro de cada una de estas actividades y su  alcance espacial, abarcando todo el territorio nacional.\\

La actividad comercial de la Red de Tiendas Minoristas de la Corporación, genera diariamente un gran volumen de información, la que necesita ser almacenada de manera eficiente. Es por ello que la Dirección General de Informática y Comunicaciones de CIMEX,  DATACIMEX, desde el año 2008 desarrolló un Almacén de Datos Operacionales, conocido en la literatura como Operational Data Store (ODS, por sus siglas en inglés), que contiene la información comercial minorista. Un ODS es un repositorio de datos operacionales a nivel de detalle, de frecuente actualización, donde se almacena la información unificada y homogeneizada proveniente de diferentes fuentes. \\

A partir del ODS desarrollado, se decidió implementar un Sistema de Administración de Negocios (SAN), con el fin de facilitar la toma de decisiones a los especialistas de la empresa al brindarle información más actualizada y centralizada. Este sistema muestra la información individual de los establecimientos a través de algunos reportes previamente diseñados y  contiene una serie de análisis comerciales de los mismos que permite tener mayor conocimiento, desde el punto de vista comercial, de cuáles son las principales características que tienen en común o cuáles no. Este análisis es importante para conocer comercialmente cómo se comportan los establecimientos hasta el momento visto.\\

Actualmente el sistema de administración presenta una serie de conceptos y funcionalidades los cuales se encuentran inutilizados o que es necesario actualizar para agregarle parte de las funcionalidades que le resultan imprescindibles a la empresa para un manejo efectivo de sus datos, siendo necesaria una reestructuración de estos, para sacarle el mayor provecho por parte de los estadistas de la empresa a los datos que este presenta y las futuras decisiones a tomar por la empresa. \\

Por otra parte, en el marco del ambiente de desarrollo de soluciones computacionales, cabe destacar que en CIMEX se trabaja con Microsoft SQL Server desde hace más de diez años. Las dificultades para obtener soporte, parches y actualizaciones de seguridad de los softwares utilizados en las empresas cubanas en la actualidad constituyen una vulnerabilidad importante. Por esta razón el país ha decidido implementar una política más decidida hacia el software libre para aminorar estas vulnerabilidades. A raíz de esto, se aprovecha la necesidad de actualización del sistema implementado actualmente para migrar de tecnología hacia Python, un poderoso lenguaje de programación que gana popularidad gracias a su legibilidad de código y las facilidades que ofrece, en conjunto con Django, un framework para desarrollo web, ambos softwares de distribución y uso libre.  
\newpage

Por tanto para este proyecto de investigación se plantea el siguiente \textbf{problema científico}

\textit{Realizar un modelado de los datos comerciales de los establecimientos de la Red de Tiendas Minoristas de la Corporación CIMEX a través de una aplicación Web dentro del  portal SAN que facilite la toma de decisiones.}\\

Para dar solución al problema anterior se plantean la siguiente \textbf{pregunta científica} 
\begin{itemize}
\item ¿Cómo desarrollar una aplicación Web dentro del portal SAN que muestre el análisis comercial de los establecimientos de la Red de Tiendas Minoristas basado en los principales indicadores comerciales, para la toma de decisiones de la empresa?
\end{itemize}

El presente proyecto de investigación se propone como \textbf{objetivo general}
\begin{itemize}
\item	Reestructurar y desarrollar  los modelos presentes en el SAN para mejorar el análisis comercial de los establecimientos de la Red de Tiendas Minoristas de la Corporación CIMEX en una aplicación Web dentro del portal SAN que facilite la toma de decisiones de la empresa.
\end{itemize}

Y como \textbf{objetivos específicos}
\begin{itemize}
\item	Identificar las herramientas de la Inteligencia de Negocios utilizadas en la Corporación CIMEX  para la toma de decisiones.
\item	Seleccionar la metodología de investigación a utilizar. 
\item	Seleccionar los indicadores comerciales que brindan mayor información sobre el comportamiento de los establecimientos de la Red de Tiendas Minoristas.
\item	Selección de los indicadores presentes en el SAN, aquellos que se desean mantener.
\item	Identificar cuáles son los nuevos conceptos a agregar al sistema.
\item	Determinar el algoritmo para el análisis de  los indicadores comerciales seleccionados. 
\item	Desarrollar una aplicación Web dentro del portal SAN que muestre el análisis comercial de los establecimientos de la Red de Tiendas Minoristas, facilitando la toma de decisiones de la Corporación CIMEX.
\end{itemize}

(Breve analisis de los capitulos de la tesis)


\chapter{Marco Teórico-Conceptual}\label{chapter:state-of-the-art}
Un negocio maneja grandes volúmenes de información de todo tipo,  la gestión de la misma es un factor esencial para su éxito. De ahí la importancia de contar con sistemas de información que permitan el control, visibilidad, orden, disposición y vinculación de todo ese movimiento de datos. A partir de estas necesidades surgen una serie de conceptos, modelos y tecnologías, cuyo estudio es necesario para poder explotar al máximo las facilidades que ofrecen. En el presente capítulo se discuten brevemente aquellos que constituyen el fundamento teórico y metodológico para el desarrollo de esta investigación.

\section*{Gestión del Conocimiento}\label{gestion_conocimiento}
\addcontentsline{toc}{section}{Gestión del Conocimiento}
Con el auge del internet las tecnologías experimentan cambios monumentales para ayudar a las organizaciones a optimizar sus procesos y realizar transacciones  comerciales de manera más eficiente. Ejemplo de ello ha sido en la tecnología de la información ya que los volúmenes de datos se encuentran en crecimiento exponencial y con estos se han desarrollado cada vez más interrogantes de como almacenarlos de manera eficiente para aprovechar e incrementar su valor.\\

La gestión del conocimiento (GC) está encargada de organizar y gestionar la información de manera eficaz, esta tiene el fin de transferir el conocimiento desde el lugar donde es generado hasta donde será asimilado, valorado, transferido y posteriormente utilizado tratando de mantener la eficiencia para maximizar y aprovechar su valor. Este es un proceso que asegura la aplicación y desarrollo de conocimientos pertinentes de una empresa para contribuir a la sostenibilidad competitiva de la misma, la solución de problemas y mantener su capacidad (Andreu \& Sieber 1999). Para desarrollarla es necesario que en la empresa madure la cultura empresarial y organizacional, se disponga de equipos humanos convenientemente preparados y se utilice una metodología de desarrollo que facilite el análisis, diseño, instrumentación, puesta en producción y control de proyectos con la disposición de herramientas informáticas. Sin estas condiciones, no es posible garantizar una acertada gestión del conocimiento.\\

Aproximadamente el 81\% de las empresas más grandes de los Estados Unidos utilizan diversas formas de GC (Becerra-Fernández \& Sabherwal, 2001), sin embargo se plantea que muchas organizaciones ha sido difícil implementarlas y mantener programas eficaces dado que la estimación de fallo oscila entre el 50\% y 70\% (Turban et al., 2005). De igual forma se indica que existen algunas experiencias positivas con respecto a este sistema ya que es practicado en el 80\% de las organizaciones alrededor del mundo. La GC no es tarea fácil para las organizaciones fácil debido a la complejidad de identificar, valorar e implementar el conocimiento pertinente para obtener ventajas competitivas, pero tampoco es una meta imposible (De Freitas \& Yaber, 2015). Una de las aplicaciones más extendidas tiene lugar en los sistemas de toma de decisiones, que generalmente se enfocan en estrategias de inteligencia empresarial. La inteligencia empresarial se ha convertido en un modelo de control y de crecimiento corporativo para lograr la sostenibilidad competitiva por su adaptabilidad a los cambios del mercado y la resolución de problemas a los clientes.

\section*{Inteligencia de Negocios}\label{BI}
\addcontentsline{toc}{section}{Inteligencia de Negocios}
hola

\section*{Modelos de datos}\label{models}
\addcontentsline{toc}{section}{Modelos de datos}
Una solución de inteligencia de negocios se alimenta de los datos transaccionales que se generan en la empresa en su accionar diario, además de datos obtenidos en fuentes externas de las que puede disponer. Los sistemas transaccionales de una organización recogen de manera sistemática grandes volúmenes de datos que describen los principales procesos del negocio, con lo cual se satisfacen las necesidades operacionales de la empresa. (Bernabeu, 2010)\\

Los datos almacenados en los sistemas transaccionales son transformados en información, dada la diversidad y la heterogeneidad de las fuentes de datos y los formatos, esta información es demasiado compleja y difícil de extraer de manera que sea efectiva. Luego, es preciso organizar, consolidar y almacenar los datos para después transformarlos y extraer conocimiento y para esto se debe determinar el modelo de datos adecuado para su almacén y futuras consultas para la extracción de este mismo conocimiento.

\section*{Modelo Multidimencional}\label{multidim_model}
\addcontentsline{toc}{section}{Modelo Multidimencional}
Un modelo de datos diseñado para el desarrollo de una BI debe sintetizar la lógica del negocio con vistas a brindar a los analistas información centralizada y resumida como apoyo a la toma de decisiones, mostrándose esta de la forma más rápida y fiable. El modelo relacional para la normalización de los datos atenta contra la simplicidad y eficiencia del proceso analítico en los sistemas transaccionales, por tanto se evidencia el surgimiento de modelos dimensionales para la trata de los datos. Contrario a lo que muchos afirman, el enfoque dimensional no fue creado por Ralph Kimball, los términos asociados a este modelo fueron producidos en los 60’s en un proyecto de investigación conjunto entre la Universidad de Dartmouth y la empresa de productos alimenticios General Mills. (Kimball, et al., 2002 p. 16)\\

Conceptualmente, este modelo define una estructura multidimensional de la información semejante a los sujetos de análisis del negocio, empleando dos componentes fundamentales: los hechos y las dimensiones. Un hecho (del inglés, fact) es un dato numérico que describe el comportamiento del negocio. Los hechos se almacenan en tablas de hechos (del inglés, fact table). Una dimensión (del inglés, dimension) es un criterio de análisis que caracteriza el negocio y permite navegar por la información localizada en las tablas de hechos. Cada dimensión es representada por una tabla en el modelo, conocida como tabla de dimensión (del inglés, dimension table). (Kimball, et al., 2002) \\

Una tabla de hechos está compuesta por medidas numéricas relacionadas con un conjunto de tablas de dimensiones. Es por ello que una tabla de hechos posee una llave primaria compuesta por llaves foráneas a las dimensiones que intervienen en el sujeto del negocio que se está modelando, expresando una relación muchos a muchos entre las dimensiones. (Kimball, et al., 2002). Estas también son vistas como las encargadas de almacenar observaciones o eventos, y pueden ser, por ejemplo, pedidos de ventas, existencias, tasas de cambio, temperaturas, etc. Una tabla de hechos contiene columnas de clave de dimensiones relacionadas con las tablas de dimensiones y columnas de medida numéricas. Las columnas de clave de dimensiones determinan la dimensionalidad de una tabla de hechos, mientras que los valores de clave de dimensiones determinan la granularidad de una tabla de hechos.\\

Las tablas de dimensiones describen entidades empresariales (las cosas que se modelan). Las entidades pueden incluir productos, personas, lugares y conceptos, incluido el propio tiempo. La tabla más coherente de un esquema de estrella es una tabla de dimensiones de fecha. Una tabla de dimensiones contiene una columna (o columnas) de clave que actúa como identificador único y columnas descriptivas.\\

Una estructura multidimensional puede ser modelada a través de dos esquemas principales:

\begin{itemize}
\item \textbf{Esquema de Estrella:} el centro de la estrella es la tabla de hechos y las puntas son las tablas de dimensiones del cubo, un ejemplo de su visualización seria el visto en la Figura 1.1
\item \textbf{Esquema de Copo de Nieve:} es una variación del esquema de estrella en la cual se normalizan algunas de las dimensiones esencialmente con el objetivo de representar las jerarquías involucradas de manera explícita.
\end{itemize}

Otra de las ventajas del modelo dimensional es la posibilidad de filtrar los datos de modo sencillo, extrayendo un subconjunto del conjunto de los datos para análisis más detallados (slicing and dicing)\\

%%
% IMAGEN
%%


El enfoque multidimensional incluye además los siguientes conceptos:
\begin{itemize}
\item \textbf{Medidas:} Las medidas o métricas son los hechos y las sumarizaciones preestablecidas que se efectúan sobre los hechos y que se crean utilizando funciones de agregación, funciones matemáticas, funciones estadísticas, operadores matemáticos y lógicos. Las medidas pueden ser aditivas, semiaditivas o no aditivas: (Kimball, et al., 2002)
\item \textbf{Atributos:} Los atributos expresan las propiedades y/o las características de las dimensiones. 
\item \textbf{Jerarquías:} Una jerarquía representa un orden parcial que se establece en una dimensión y que se expresa mediante una estructura arbórea. La principal ventaja de establecer relaciones jerárquicas en una dimensión reside en la navegación por los datos al desplazarse en profundidad por el árbol correspondiente.
\end{itemize}

El rendimiento de un sistema de gestión de bases de datos (DBMS, Database Management System) está directamente relacionado con la eficiencia del almacenamiento de datos en disco y su movimiento hacia los registros de CPU para el procesamiento. El modelo dimensional fue instrumentado exitosamente en la época de los servidores de 32 bits, con uno o dos procesadores y menos de un gigabyte de RAM, cuando el almacenamiento por filas era la única opción para las bases de datos (Russo et al., 2012). \\

En búsqueda de la eficiencia de estos procesos, ha crecido el interés en el almacenamiento orientado a columnas (column-oriented storage). En este tipo de distribución cada página de datos contiene los valores correspondientes a una sola columna y, al tratarse de datos estructurados, en el proceso de indización se conservan los valores repetidos solo una vez, todo lo cual favorece la compresión y rápida recuperación. Las comparaciones entre las bases de datos orientadas a filas y a columnas suelen estar relacionadas con la eficiencia del acceso al disco duro para una carga de trabajo determinada, ya que el tiempo de búsqueda es increíblemente largo en comparación con otros cuellos de botella en las computadoras.


\section*{Python y Django}\label{py_dj}
\addcontentsline{toc}{section}{Python y Django}
python y django




\chapter{Grupo Empresarial CIMEX: gestión comercial}\label{chapter:proposal}

En toda empresa para mantener un rendimiento ascendente es de vital importancia que los directivos y analistas presenten un acceso cómodo y fiable a información útil y valiosa referente al desempeño del negocio. La base de este tipo de sistema de información son los datos recuperados, conciliados, almacenados, posteriormente analizados y convertidos en conocimiento, todo lo cual debe contribuir de forma satisfactoria a la toma de decisiones en la empresa.\\

El Grupo Empresarial CIMEX es una de las principales entidades que presenta el país dedicadas al comercio y la exportación, es considerada entre las primeras organizaciones en utilizar soluciones computacionales con el fin de favorecer la toma de decisiones. En el presente capitulo se explica brevemente el funcionamiento de el negocio, las aplicaciones que se han implementado hasta el momento para la toma de decisiones, esencialmente en términos de la información comercial, así como la descripción del escenario actual que propicia el desarrollo de la presente investigación en función de lograr una nueva aproximación de solución de inteligencia de negocios.


\section*{Grupo Empresarial CIMEX}\label{CIMEX}
\addcontentsline{toc}{section}{Grupo Empresarial CIMEX}
La Corporación de Exportación e Importación (CIMEX) es una sociedad de capital estatal cubano compuesta por numerosas entidades cuyo principal objetivo es la adquisición y la comercialización de productos y servicios. Esta maneja un amplio conjunto de líneas de negocio que comprende desde la comercialización de forma minorista y mayorista hasta otros relacionados con la importación y la exportación de mercancías y el transporte marítimo, así como los servicios de gastronomía, tecnológicos y de aduana.\\

La corporación se encuentra estructurada en vicepresidencias y direcciones, sucursales territoriales, sociedades afiliadas y divisiones especializadas. Una de las divisiones especializadas es la División DATACIMEX, encargada de poner a disposición de los usuarios de CIMEX una amplia gama de aplicaciones computacionales que facilitan sus actividades. Hace más de diez años en DATACIMEX se comenzó a trabajar en la transformación de los datos primarios en función de la toma de decisiones.\\

El grupo de Inteligencia de Negocios, perteneciente a esta División, ha desarrollado desde el año 2008 un portal web para el apoyo a la toma de decisiones, denominado Sistema de Administración de Negocios (SAN). En SAN se muestran reportes estáticos sobre diferentes procesos de negocio que se interrelacionan en la organización, brindando a los directivos un conjunto de aplicaciones que abarcan desde la etapa de planificación y ejecución de cada uno de los procesos hasta la evaluación de comportamientos mediante indicadores, cuadros de mando y emisión automática de boletines.



\section*{Gestión Comercial relacionada con las Estadísticas Minoristas}\label{gestion_comercial}
\addcontentsline{toc}{section}{Gestión Comercial relacionada con las Estadísticas Minoristas}
Una de las actividades fundamentales de la corporación es el comercio minorista. Las entidades pertenecientes a este comercio son también conocidas como entidades. Esta red de establecimientos  minoristas está conformada por: Las tiendas Panamericanas, los Servicentros Cupet y Oro Negro, las cafeterías El Rápido, los videocentros, las tiendas fotográficas Photoservice, los centros comerciales y las pequeñas tiendas en los barrios, denominadas puntos de venta.\\

La red está organizada de forma jerárquica teniendo en su nivel más alto a las Sucursales, las cuales se dividen en Centros Contables o Complejos, los que a su vez se subdividen en Establecimientos Independientes, formados por Establecimientos Dependientes (Figura 2.1).

%%
% IMAGEN
%%

La actividad comercial de las tiendas minoristas genera diariamente gran cantidad de información la cual es necesaria mantener actualizada y administrada de forma fiable para lograr un mejor desempeño para las analistas y directivos de la corporación y también otras entidades pertenecientes al país como son la Oficina Nacional de Estadísticas e Información (ONEI) y el Ministerio de Comercio Interior (MINCIN). Desde la perspectiva comercial se realizan varias operaciones que provocan movimientos de entrada y salida en el inventario relacionadas con los conceptos compra y venta de mercancías, transferencias y ajustes. Es necesario mantener un seguimiento periódico del comportamiento de estas operaciones, pues con un análisis certero y previsor se evitan posibles desabastecimientos o sobre-inventarios y es posible aumentar el nivel de las ventas y la satisfacción del cliente. Estos comportamientos se analizan a partir de un conjunto de indicadores comerciales.


\section*{Conceptos Comerciales}\label{concepto_comercial}
\addcontentsline{toc}{section}{Conceptos Comerciales}
En un establecimiento de la red minorista de CIMEX se realizan varias actividades comerciales como venta de mercancías, servicios fotográficos, venta de combustible y servicios de gastronomía. Además existen diferentes áreas o localidades, entre las que se encuentran almacén, merma y venta. Para mantener correctos los niveles de inventario en las tiendas y en los almacenes centrales es necesario efectuar el seguimiento sistemático del comportamiento del negocio en relación con las ventas.\\

Los conceptos en el ámbito comercial manejados por CIMEX son:

\begin{itemize}
\item \textbf{Compras:} Representan el acto mediante el cual CIMEX, como sujeto económico, entra en posesión de un bien o servicio mediante el pago del precio a un proveedor. (La Gran Enciclopedia de Economía, 2009) 
\item \textbf{Ventas:} Representan las entregas a clientes de productos terminados, trabajos efectuados, servicios prestados y mercancías. (Grupo Empresarial CIMEX, 1999) 
\item \textbf{Inventario:} Es el conjunto de mercancías o artículos acumulados en el almacén en espera de ser vendidos o utilizados en el proceso productivo. (La Gran Enciclopedia de Economía, 2009) Es decir, representa la existencia física de todos los productos en un establecimiento. 
\item \textbf{Transferencias:} Responden a los movimientos de mercancías entre establecimientos o entre localidades de un mismo establecimiento, por ejemplo, del almacén al área de ventas. Un movimiento de mercancía se registra en los sistemas de control como una transferencia de salida en el origen y una transferencia de entrada en el destino.
\item \textbf{Ajustes:} Engloban todas las operaciones que se realizan en el inventario de un establecimiento para reflejar los niveles de existencia reales. Por ejemplo, los faltantes y sobrantes en un establecimiento, que se detectan en el conteo físico, deben ser ajustados en el sistema de control, sea o no computacional, para que quede reflejado en el inventario real del establecimiento. Otros motivos para realizar ajustes son los errores que se cometen al registrar una operación de venta o transferencia y la afectación del inventario por rotura o deterioro de los productos.
\item \textbf{Vales:} Responden a la entrada o la salida del inventario de mercancías mediante otro concepto que no sea Compra o Venta. El vale de entrada representa la recepción de mercancía que no fue comprada a un proveedor, por ejemplo, una donación. El vale de entrega representa la salida del inventario de mercancía que no fue vendida sino que se traslada del establecimiento a otras entidades que no pertenecen a CIMEX, previo acuerdo entre las partes. 
\end{itemize}  


\section*{Soluciones Precedentes}\label{sol_presed}
\addcontentsline{toc}{section}{Soluciones Precedentes}
El análisis de las Estadísticas Minoristas en la corporación ha pasado en el decursar de los años por diferentes soluciones para la concepción y análisis de la información mediante los estadistas. Se ha priorizado tanto la automatización de sus principales operaciones empresariales como la evolución necesaria de acuerdo con el desarrollo científico, tecnológico y profesional en sus entidades. No solo se han dado respuestas oportunas y pertinentes al procesamiento de los datos operacionales sino también se ha trabajado con el propósito de facilitar los procesos de dirección con herramientas y ambientes computacionales propios, que han sentado las bases para el desarrollo de la presente investigación.\\

Con anterioridad la empresa contaba con diferentes sistemas computacionales encargados de la recolección de la informacion ERP (Enterprise Resource Planning) los cuales eran: Silver, Sentai-Trax y Sentai-Viper. Por razones ajenas a la investigación de la presente tesis en la actualidad se cuenta con un único ERP conocido como Sentai el cual presenta la información que será utilizada por los analistas y directivos.\\

En el año 2000 se desarrolló en CIMEX una solución implementada en el lenguaje de programación Visual Foxpro 5.0, que brindaba funcionalidades para el análisis de los indicadores comerciales minoristas. Consistía en una aplicación de escritorio instalada en cada una de las sucursales y una versión central en la dirección de CIMEX (Figura 2.3). Esta primera aproximación logró resolver las necesidades básicas de información, entre ellas la presentación de ciertos indicadores comerciales como la utilidad de la venta de un producto en un mes o la acumulada en el año actual. Los principales resultados que produjo este sistema fueron: 

\begin{itemize}
\item Ubicó a CIMEX entre las primeras organizaciones cubanas en diseñar una solución computacional en términos de la toma de decisiones. 
\item Constituyó la primera solución que unificaba los datos disgregados en los diferentes sistemas operacionales, permitiendo el seguimiento de los indicadores comerciales.
\item Logró brindar un conjunto de funcionalidades básicas para el análisis y la toma de decisiones, tales como visualizar reportes parametrizados de los principales indicadores comerciales.
\item El lenguaje de programación Visual Foxpro 5.0 se convirtió en obsoleto, lo que dificultó la realización de actualizaciones.
\item La interfaz visual no era intuitiva.  
\item En una misma consulta no era posible realizar comparaciones entre diferentes años. 
\item Los mecanismos de sincronización entre las diferentes instancias de la aplicación no eran suficientes y se limitaba la actualización de los datos.
\end{itemize}



\include{MainMatter/Proposal}
\include{MainMatter/Implementation}

\backmatter

\include{BackMatter/Conclusions}
\include{BackMatter/Recomendations}
\section*{Bibliografía}
\printbibliography[heading=bibintoc]


Abadi, D., Boncz, P., Harizopoulos, S., Idreos, S., y Madden, S. (2012). The Design and Implementation of Modern Column-Oriented Database Systems. Foundations and Trends in Databases, 5(3): 197–280.\\

Abadi, D. J., Madden, S. R., y Hachem, N. (2008). Column-Stores vs. Row-Stores: How Different Are They Really? Trabajo presentado en 28th ACM SIGMOD/PODS International Conference on Management of Data / Principles of Database Systems, Vancouver, BC, Canada.\\

Kimball, Ralph and Ross, Margy. 2002. The data warehouse toolkit : the complete guide to dimensional modeling. Second Edition. s.l. : Wiley Computer Publishing, 2002. ISBN 0-471- 20024-7\\

La Gran Enciclopedia de Economía. 2009. Definición de compra. La Gran Enciclopedia de Economía. [En línea] Economía48.com, 2009. [Citado el: 18 de 10 de 2012.] http://www.economia48.com/spa/d/compra/compra.htm.
\begin{itemize}
\item 2009. La Gran Enciclopedia de Economía. [En línea] Economía48.com, 2009. [Citado el: 18 de 10 de 2013.] http://www.economia48.com/spa/d/inventario/inventario.htm.
\end{itemize}

Russo, M., Ferrari, A., y Webb, C. (2012). Microsoft SQL Server 2012 Analysis Services: The BISM Tabular Model. United States of America: O’Reilly\\

Turner, Hammond, y Cotton. (1979). A DBMS for Large Statistical Database. Trabajo presentado en VLDB Conference (Very Large Data Base Endowment), Rio de Janeiro, Brazil.\\

Vitt, Liz y Cameron, Scott. 2012. Choosing a Tabular or Multidimensional Modeling Experience in SQL Server 2012 Analysis Services. [En línea] 2012. [Citado el: 22 de Noviembre de 2012.] http://msdn.microsoft.com/en-us/library/hh994774.aspx



\end{document}