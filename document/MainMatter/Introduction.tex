\chapter*{Introducción}\label{chapter:introduction}
\addcontentsline{toc}{chapter}{Introducción}


Los múltiples avances tecnológicos y técnicos, como herramientas para el desarrollo del conocimiento, han permitido a las organizaciones el manejo inteligente y óptimo de sus recursos, les ha dado acceso a elementos y prácticas innovadoras que permiten poner en marcha estrategias y acciones efectivas para manejar sus negocios de manera organizada y mantener un riguroso control operacional. Las empresas cubanas promueven la explotación de estas ventajas brindadas por el ascenso tecnológico.\\

La Corporación CIMEX, es un grupo empresarial, representado  a lo largo de todo el territorio nacional, con actividades en importantes sectores de la economía cubana, estructurado en 19 sucursales territoriales, más de 80 empresas, 10 subsidiarias y 13 divisiones especializadas comprenden amplios e importantes sectores de la economía cubana. Una de sus estructuras más significativa es la Red de Tiendas Minoristas, integrada por las Tiendas Panamericanas, los Servicentros “Servicupet”, las Cafeterías “El Rápido”, los Videocentros, las Tiendas Fotográficas “Photoservice”, los Centros Comerciales y pequeñas tiendas típicas en barrio (Corporación CIMEX - Ecured, 2012).\\

La Corporación CIMEX surge en el año 1979, con la creación de su primera división, Havanatur, encargada del movimiento de personas entre Cuba y Estados Unidos. En una segunda fase de desarrollo, comprendida  hasta 1992, crea otras divisiones con actividades de perfil comercial, fundamentalmente, de carácter mayorista. En 1993, a partir de la despenalización de la tenencia de divisas, el auge del turismo internacional y el inicio de la recuperación económica del país, CIMEX inicia una etapa de expansión al sector de los servicios minoristas, ampliándose su objeto social, que comprende hoy,  la realización de actividades productivas, de carácter comercial mayorista, minorista y de servicios, la exportación y la importación; constituyendo su rasgo más significativo el amplio espectro de cada una de estas actividades y su  alcance espacial, abarcando todo el territorio nacional.\\

La actividad comercial de la Red de Tiendas Minoristas de la Corporación, genera diariamente un gran volumen de información, la que necesita ser almacenada de manera eficiente. Es por ello que la Dirección General de Informática y Comunicaciones de CIMEX,  DATACIMEX, desde el año 2008 desarrolló un Almacén de Datos Operacionales, conocido en la literatura como Operational Data Store (ODS, por sus siglas en inglés), que contiene la información comercial minorista. Un ODS es un repositorio de datos operacionales a nivel de detalle, de frecuente actualización, donde se almacena la información unificada y homogeneizada proveniente de diferentes fuentes. \\

A partir del ODS desarrollado, se decidió implementar un Sistema de Administración de Negocios (SAN), con el fin de facilitar la toma de decisiones a los especialistas de la empresa al brindarle información más actualizada y centralizada. Este sistema muestra la información individual de los establecimientos a través de algunos reportes previamente diseñados y  contiene una serie de análisis comerciales de los mismos que permite tener mayor conocimiento, desde el punto de vista comercial, de cuáles son las principales características que tienen en común o cuáles no. Este análisis es importante para conocer comercialmente cómo se comportan los establecimientos hasta el momento visto.\\

Actualmente el sistema de administración presenta una serie de conceptos y funcionalidades los cuales se encuentran inutilizados o que es necesario actualizar para agregarle parte de las funcionalidades que le resultan imprescindibles a la empresa para un manejo efectivo de sus datos, siendo necesaria una reestructuración de estos, para sacarle el mayor provecho por parte de los estadistas de la empresa a los datos que este presenta y las futuras decisiones a tomar por la empresa. \\

Por otra parte, en el marco del ambiente de desarrollo de soluciones computacionales, cabe destacar que en CIMEX se trabaja con Microsoft SQL Server desde hace más de diez años. Las dificultades para obtener soporte, parches y actualizaciones de seguridad de los softwares utilizados en las empresas cubanas en la actualidad constituyen una vulnerabilidad importante. Por esta razón el país ha decidido implementar una política más decidida hacia el software libre para aminorar estas vulnerabilidades. A raíz de esto, se aprovecha la necesidad de actualización del sistema implementado actualmente para migrar de tecnología hacia Python, un poderoso lenguaje de programación que gana popularidad gracias a su legibilidad de código y las facilidades que ofrece, en conjunto con Django, un framework para desarrollo web, ambos softwares de distribución y uso libre.  
\newpage

Por tanto para este proyecto de investigación se plantea el siguiente \textbf{problema científico}

\textit{Realizar un modelado de los datos comerciales de los establecimientos de la Red de Tiendas Minoristas de la Corporación CIMEX a través de una aplicación Web dentro del  portal SAN que facilite la toma de decisiones.}\\

Para dar solución al problema anterior se plantean la siguiente \textbf{pregunta científica} 
\begin{itemize}
\item ¿Cómo desarrollar una aplicación Web dentro del portal SAN que muestre el análisis comercial de los establecimientos de la Red de Tiendas Minoristas basado en los principales indicadores comerciales, para la toma de decisiones de la empresa?
\end{itemize}

El presente proyecto de investigación se propone como \textbf{objetivo general}
\begin{itemize}
\item	Reestructurar y desarrollar  los modelos presentes en el SAN para mejorar el análisis comercial de los establecimientos de la Red de Tiendas Minoristas de la Corporación CIMEX en una aplicación Web dentro del portal SAN que facilite la toma de decisiones de la empresa.
\end{itemize}

Y como \textbf{objetivos específicos}
\begin{itemize}
\item	Identificar las herramientas de la Inteligencia de Negocios utilizadas en la Corporación CIMEX  para la toma de decisiones.
\item	Seleccionar la metodología de investigación a utilizar. 
\item	Seleccionar los indicadores comerciales que brindan mayor información sobre el comportamiento de los establecimientos de la Red de Tiendas Minoristas.
\item	Selección de los indicadores presentes en el SAN, aquellos que se desean mantener.
\item	Identificar cuáles son los nuevos conceptos a agregar al sistema.
\item	Determinar el algoritmo para el análisis de  los indicadores comerciales seleccionados. 
\item	Desarrollar una aplicación Web dentro del portal SAN que muestre el análisis comercial de los establecimientos de la Red de Tiendas Minoristas, facilitando la toma de decisiones de la Corporación CIMEX.
\end{itemize}

(Breve analisis de los capitulos de la tesis)

